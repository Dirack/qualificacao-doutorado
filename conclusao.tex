\chapter{CONCLUSÃO}
\label{cap10:conclusao}

As etapas já concluídas demonstram a que o método ERC pode ser utilizado para a obtenção da seção empilhada sem a necessidade
de informação a priori sobre o modelo de velocidades ou de uma etapa de análise de velocidades. Para melhor corroborar esta 
afirmação é necessário reproduzir o exeperimento aqui descrito com outros modelos iniciais, além do modelo do refletor gaussiano
aqui utilizado, e obter a seção de afastamento nulo.
Da forma como o algoritmo foi produzido, este deve ser capaz de produzir a seção empilhada mesmo para modelos de múltiplos refletores
com distribuições mais complexas de velocidade.

Nos resultados do empilhamento ERC no Capítulo \ref{cap7:empilhamento}, observamos que a forma do evento de reflexão na seção
de afastamento nulo foi preservada e as amplitudes foram realçadas. Porém, para testar como o algoritmo reage
a dados com ruído, é necessário reproduzir o experimento com diversos níveis de ruído aleatório adicionado aos dados modelados.

Outra observação é a de que o processo de empilhamento ERC adiciona ruído numérico aos traços da seção empilhada. Uma forma de
atenuar este ruído seria buscar uma maior precisão na determinação dos traços das famílias ERC: Aumentando a precisão das variáveis
nos programas (passando do tipo \textit{float} para \textit{double}, por exemplo), e realizando a interpolação e regularização
mais de uma vez nos dados modelados até que a discretização seja satisfatória.

Enfim, os resultados deste trabalho serão publicados em dois artigos científicos: O primeiro baseado 
nos resultados obtidos e e os que 
serão obtidos da repetição deste experimento para modelos mais complexos, este irá focar no método ERC e na obtenção da seção
empilhada ERC a partir da aplicação da condição SDC à aproximação não hiperbólica do tempo de trânsito SRC (Equação \ref{eq:2.4}).
O segundo artigo irá tratar da inversão do modelo de velocidades a partir da metodologia de inversão aqui proposta
no Capítulos 9 a 11.

As etapas já concluídas demonstram que a estratégia de inversão do modelo de velocidades é promissora,
pois permite localizar corretamente as fontes pontuais PIN sobre os refletores em profundidade
e obter um modelo de velocidades suavizado que pode ser utilizado como primeiro passo
para a construção do modelo de velocidades em profundidade. Todavia,
esta estratégia necessita de uma melhor forma de representar o modelo de velocidades com variação lateral
de velocidades e metodologia de interpolação.

O critério de convergência utilizado também precisa ser alterado para poder tratar modelos mais complexos.
Neste relatório utilizamos as diferenças entre o tempo de trânsito obtido no traçamento de raios e a fórmula
do ERC. Porém, a fórmula do ERC depende dos parâmetros $R_{NIP}$ e $\beta_0$ obtidos durante o empilhamento ERC.
A incerteza na determinação destes parâmetros resulta em incerteza na determinação da localização das fontes
pontuais PIN durante a inversão e na determinação dos tempos de trânsito com a fórmula ERC. Por isso, seria
melhor utilizar o semblance calculado nas famílias ERC nos dados pré-empilhados, ajustando as curvas de tempo
de trânsito obtidas no traçamento de raios aos dados.

Por fim, para cumprir os objetivos estabelecidos no cronograma deste relatório solicito a prorrogação do prazo de defesa da minha tese de doutorado por mais um ano além do prazo normal de defesa. Os motivos do atraso no 
desenvolvimento da tese foram explicitados na justificativa e oque foi desenvolvido desde a interrupção
das atividades laboratoriais por causa da pandemia de Covid-19 também foi devidamente apresentado neste relatório.
